\documentclass{article}
\usepackage[utf8]{inputenc}
\usepackage{amsmath}
\usepackage{amsfonts}
\usepackage{amssymb}

\begin{document}
\section*{Oct 14 notes}
\textbf{Activity 2.6.2:}\\
$\vec{e}_{1} = \hat{i} = \begin{bmatrix} 
1\\ 
0\\
0\\
\end{bmatrix}$, $\vec{e}_{2} = \hat{j} =\begin{bmatrix} 
0\\ 
1\\
0\\
\end{bmatrix}$, $\vec{e}_{3} = \hat{k} =\begin{bmatrix} 
0\\ 
0\\
1\\
\end{bmatrix}$\\
\\
a) $\vec{v}$ can be expressed as a linear combination of $\vec{e}_{1}$, $\vec{e}_{2}$, and $\vec{e}_{3}$ because we have a pivot in every row, we can set i, j, and k to be any scalar to get any values of $\vec{v}$ \\
\\
b) If $\vec{w} = \begin{bmatrix} 
1\\ 
1\\
0\\
\end{bmatrix}$, $\vec{v}$ can't be expressed as a linear combination of $\vec{e}_{1}$, $\vec{e}_{2}$, and $\vec{w}$ because we don't have a pivot in row 3, so not spanning set. \\
\\
c) All vectors in $\mathbb{R}^{3}$ can be written as linear combinations of $\vec{e}_{1}$, $\vec{e}_{2}$, and $\vec{e}_{3}$  because we have a pivot in every row and column - linearly independent vectors that span the set\\
\\
\textbf{Observation 2.6.3:} A basis is a linearly independent set that spans a vector space. In example 2.6.2, $\vec{e}_{1}$, $\vec{e}_{2}$, and $\vec{e}_{3}$ is an example of the standard basis vectors of $\mathbb{R}^{3}$.\\
\\
\textbf{Observation 2.6.4} A basis may be thought of as a collection of building blocks for a vector space, since every vector in the space can be expressed as a unique linear combination of basis vectors.\\
\\
\textbf{Observation 2.5.8 says:}\\
- A set of vectors is \textbf{linearly independent} if and only if the RREF has all pivot columns.\\
- A set of $\mathbb{R}^{m}$ vectors \textbf{spans $\mathbb{R}^{m}$} if the RREF has all pivot rows\\
\\
\textbf{Definition 2.6.3 says:}\\
- A basis is a linearly independent set that spans a vector space. \\
(This means it should have pivots in every row and column.)\\
\\
\textbf{Activity 2.6.5}\\
Label each of the sets as: spanning $\mathbb{R}^{4}$, linearly independent, or a basis for $\mathbb{R}^{4}$
\\
$
A = \begin{bmatrix} 
1 & 0 & 0 & 0 \\ 
0 & 1 & 0 & 0 \\
0 & 0 & 1 & 0 \\
0 & 0 & 0 & 1 \\
\end{bmatrix}
$, $
B = \begin{bmatrix} 
2 & 2 & 4 & -3 \\ 
3 & 0 & 3 & 0 \\
0 & 0 & 0 & 1 \\
-1 & 3 & 2 & 3 \\
\end{bmatrix}
$, $
C = \begin{bmatrix} 
2 & 2 & 3 & -1 & 4 \\ 
3 & 0 & 13 & 10 & 3 \\ 
0 & 0 & 7 & 7 & 0 \\ 
-1 & 3 & 16 & 14 & 2 \\ 
\end{bmatrix}
$\\$D = \begin{bmatrix} 
2 & 4 & -3 & 3 \\ 
3 & 3 & 0 & 6 \\
0 & 0 & 1 & 1 \\
-1 & 2 & 3 & 5 \\
\end{bmatrix}
$, $E = \begin{bmatrix} 
5 & -2 & 4\\ 
3 & 1 & 5\\
0 & 0 & 1\\
-1 & 3 & 3\\
\end{bmatrix}
$\\
\\

\begin{verbatim}
B=Matrix(QQ,[ [2,2,4,-3],[3,0,3,0],[0,0,0,1], [-1,3,2,3] ])
B.rref()
C=Matrix(QQ,[ [2,2,3,-1,4],[3,0,13,10,3],[0,0,7,7,0], [-1,3,16,14,2] ])
C.rref()
D=Matrix(QQ,[ [2,4,-3,3],[3,3,0,6],[0,0,1,1], [-1,2,3,5] ])
D.rref()
E=Matrix(QQ,[ [5,-2,4],[3,1,5],[0,0,1], [-1,3,3] ])
E.rref()
\end{verbatim}

$
RREF(A) = \begin{bmatrix} 
1 & 0 & 0 & 0 \\ 
0 & 1 & 0 & 0 \\
0 & 0 & 1 & 0 \\
0 & 0 & 0 & 1 \\
\end{bmatrix}
$, $
RREF(B) = \begin{bmatrix} 
1 & 0 & 1 & 0 \\ 
0 & 1 & 1 & 0 \\
0 & 0 & 0 & 1 \\
0 & 0 & 0 & 0 \\
\end{bmatrix}
$ \\$
RREF(C) = \begin{bmatrix} 
1 & 0 & 0 & -1 & 1 \\ 
0 & 1 & 0 & -1 & 1 \\ 
0 & 0 & 1 & 1 & 0 \\ 
0 & 0 & 0 & 0 & 0 \\ 
\end{bmatrix}
$\\$RREF(D) = \begin{bmatrix} 
1 & 0 & 0 & 0 \\ 
0 & 1 & 0 & 0 \\
0 & 0 & 1 & 0 \\
0 & 0 & 0 & 1 \\
\end{bmatrix}
$, $RREF(E) = \begin{bmatrix} 
1 & 0 & 0\\ 
0 & 1 & 0\\
0 & 0 & 1\\
0 & 0 & 0\\
\end{bmatrix}
$\\
\\
Span (all pivot rows in the RREF): A, D\\
Linearly Independent (RREF has all pivot columns): A, D, E\\
Basis (all pivot rows and columns in the RREF): A, D\\
None of the above: B, C
\\
\\
\textbf{Activity 2.6.6}\\
\\
If $\left\{ \vec{v}_{1}, \vec{v}_{2}, \vec{v}_{3}, \vec{v}_{4}  \right\}$ is a basis for $\matbbb{R}^{4}$ that means RREF would be:\\
\begin{bmatrix} 
1 & 0 & 0 & 0 \\ 
0 & 1 & 0 & 0 \\
0 & 0 & 1 & 0 \\
0 & 0 & 0 & 1 \\
\end{bmatrix}\\
\\
This would be the identity matrix, with pivots in all rows and columns.\\
\\
\textbf{Fact 2.6.7}\\
A basis for $R^{n}$ must have n vectors and its square matrix must reduce to the identity matrix containing all 0s except for 1 along the diagonal pointing down and to the right.\\
\\
\textbf{Observation 2.7.1} - subspace of a vector space is a subset that is itself a vector space.\\
\\
One easy way to construct a subspace is to take the span of set, but a linearly dependent set contains “redundant” vectors. For example, only two of the three vectors in the following image are needed to span the planar subspace.\\
\\
\\
\\
\textbf{Activity 2.7.2} Consider the subspace of $\mathbb{R}^{4}$ given by:\\
\\
$
W = span =  
\left\{ 
\begin{bmatrix} 
2\\ 3\\ 0\\ 1\\
\end{bmatrix},
\begin{bmatrix} 
2\\ -3\\ 2\\-3\\ 
\end{bmatrix},
\begin{bmatrix} 
2\\ -3\\ 2\\ -3\\
\end{bmatrix},
\begin{bmatrix} 
1\\ 5\\ -1\\ 0\\
\end{bmatrix} 
\right\}$
\\

(a) Mark the parts of the RREF of the matrix to show what parts show that W's spanning set is linearly dependent.

\begin{verbatim}
W=Matrix(QQ,[ [2,2,2,1],[3,0,-3,5],[0,1,2,-1], [1,-1,-3,0] ])
W.rref()
\end{verbatim}
\\
RREF(W):\\
\\
\begin{bmatrix} 
1 & 0 & -1 & 0 \\ 
0 & 1 & 2 & 0 \\
0 & 0 & 0 & 1 \\
0 & 0 & 0 & 0 \\
\end{bmatrix}\\
\\
Linearly Independent means RREF has all pivot columns, but column 3 does not have a pivot so this is a linearly dependent system. \\
\\
(b) Find a basis for W by removing a vector from its spanning set to make it linearly independent. \\
\\
Drop 3rd vector because it isn't a pivot column.\\
\begin{verbatim}
W=Matrix(QQ,[ [2,2,1],[3,0,5],[0,1,-1], [1,-1,0] ])
W.rref()
\end{verbatim}
RREF(W):\\
\\
\begin{bmatrix} 
1 & 0 & 0 \\ 
0 & 1 & 0 \\
0 & 0 & 1 \\
0 & 0 & 0 \\
\end{bmatrix}\\
\\
\textbf{Fact 2.7.3:} To compute a basis for the subspace, simply remove the vectors corresponding to the non-pivot columns of the RREF.
\end{document}