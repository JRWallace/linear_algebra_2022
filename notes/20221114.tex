\documentclass{article}
\usepackage[utf8]{inputenc}
\usepackage{amsmath}
\usepackage{amsfonts}
\usepackage{amssymb}

\begin{document}
\section*{Nov 14 notes}\\
\\
\noindent\textbf{Activity 4.3.1}\\
Let $T: \mathbb{R}^n \to \mathbb{R}^m$ be a linear map with standard matrix A. Sort the following items into three groups of statements: a group that means $T$ is injective, a group that means $T$ is surjective, and a group that means is $T$ bijective.\\
\\
Note from class: $\vec x$ can be thought of as an n x 1 matrix.\\
\\
$T(\vec{x}) = A\vec{x}$ so can act as if I applied the transformation T to $\vec{x}$.\\
\\
A. $A\vec x=\vec b$ has a solution for all $\vec b\in\mathbb{R}^m$\\
B. $A\vec x=\vec b$ has a unique solution for all $\vec b\in\mathbb{R}^m$\\
C. $A\vec x=\vec b = 0$ has a unique solution.\\
D. The columns of A span $\mathbb{R}^m$\\
E. The columns of A are linearly independent\\
F. The columns of A are a basis of Rm\\
G. Every column of RREF(A) has a pivot\\
H. Every row of RREF(A) has a pivot\\
I. m = n and RREF(A) = I\\
\\
- A set of vectors are linearly independent if their RREF matrix has all pivot columns (Observation 2.5.8). RREF of the injective map’s standard matrix has all pivot columns (Observation 3.4.7)\\
- A set of vectors spans the space if the RREF matrix has a pivot in every row (Observation 2.5.8). RREF of the surjective map’s standard matrix has a pivot in each row (Observation 3.4.7)\\
\\
C is injective bc you can only get the 0 vector via trivial solution\\
If the co-domain spans the space, we can get back any b in the domain and we have a pivot in every row and are surjective so A\\
If we have a unique solution then we are linearly independent and span, so B is bijective.\\
\\
Bijective: F,I,B \\
Injective: E,G,C \\
Surjective: D,H,A \\
\\
\newpage \noindent\textbf{Activity 4.3.2}\\
\\
Let $T: \mathbb{R}^3 \to \mathbb{R}^3$ be the linear transformation given by the standard matrix $A = \begin{bmatrix} 
2 & -1 & 0 \\ 
2 & 1 & 4 \\
1 & 1 & 3 
\end{bmatrix}$ Write an augmented matrix representing the system of equations given by $T(\vec{x}) = \vec{0}$, that is $A\vec{x} = \begin{bmatrix} 0\\0\\0 \end{bmatrix}$, then solve to find the kernel of T.\\
\\
\\
$T\left(\begin{bmatrix} x\\y\\z \end{bmatrix}\right)$ would be\\
$\begin{bmatrix}
2x & -1y & 0z \\ 
2x & 1y & 4z \\
1x & 1y & 3z 
\end{bmatrix}$ \\
Is there a non-trivial way to get the zero vector?\\
$\begin{bmatrix}
2 & -1 & 0 & 0\\ 
2 & 1 & 4 & 0\\
1 & 1 & 3 & 0 
\end{bmatrix}$
RREF:
\begin{verbatim}
Matrix(QQ,[ 
[2,-1,0,0],
[2,1,4,0],
[1,1,3,0]
]).rref()
[1 0 1 0]
[0 1 2 0]
[0 0 0 0]
\end{verbatim}
\\
Separate out the x, y, z\\
\\
$x = \begin{bmatrix} 1 \\ 0 \\ 0 \end{bmatrix}$,
$y = \begin{bmatrix} 0 \\ 1 \\ 0 \end{bmatrix}$,
$z = \begin{bmatrix} 1 \\ 2 \\ 0 \end{bmatrix}$\\
\\
means \\
1x + 0y + 1z = 0, or x = -z \\
0x + 1y + 2z = 0, or y = -2z\\
\\
$ket(T) = span\left\{\begin{bmatrix} -1 \\ -2 \\ 1 \end{bmatrix}\right\}$
\\
Definition 4.4.3, Says that if there' sa unique soution to the eqn ax=b, then its bijection\\
so there's only 1 x that works
set the inverse of t to be the exact x
A inverse is that inverse matrix of A, so A is invertible\\
\\
Activity 4.3.4\\
Let $T: \mathbb{R}^3 \to \mathbb{R}^3$ be the linear transformation given by the standard matrix $A = \begin{bmatrix} 
2 & -1 & 6 \\ 
2 & 1 & 3 \\
1 & 1 & 4 
\end{bmatrix}$ Write an augmented matrix representing the system of equations given by $T(\vec{x}) = \vec{e}_1$, that is $A\vec{x} = \begin{bmatrix} 1\\0\\0 \end{bmatrix}$, then solve $T(\vec{x}) = \vec{e}_1$ to find $T^{-1}(\vec{e}_1)$\\
\\
Create an augmented matrix while solving for the standard basis vectors, then go from there\\
\\

$\begin{bmatrix} 
2 & -1 & 6 & 1\\ 
2 & 1 & 3 & 0\\
1 & 1 & 4 & 0
\end{bmatrix}$

$T(\vec{e_1}) = T\left(\begin{bmatrix} 1\\0\\0 \end{bmatrix}\right) = $

\begin{verbatim}
Matrix(QQ,[ 
[2,-1,6,1],
[2,1,3,0],
[1,1,4,0]
]).rref()
[1 0 1 0]
[0 1 2 0]
[0 0 0 0]
\end{verbatim}

$\begin{bmatrix} 1 \\ -5 \\1\end{bmatrix}$
$\begin{bmatrix} -2 \\ 14 \\-3\end{bmatrix}$
$\begin{bmatrix} 3 \\ 18 \\4\end{bmatrix}$
\\
observation 4.3.5 - row reduce entire matrix at once \\
so 1, -5, 1 is is something 
\\
activity 4.3.6 - set up the rref and the matrix on the right is the a inverse\\
\\
activity 4.3.7  - is this invertible? give your reasons\\
must be square and must row-reduce to I.\\
\\
we must cover the whole co-domain and the element in the domain and co-domain must have a 1-1 unique counterpart\\
\\
A transformation must be bijective for its matrix to be invertible.\\
The RREF of the matrix must have all pivot rows and columns, thus also be a basis.\\
To find the inverse, set up an augmented matrix where you set the matrix equal to the standard basis vectors and solve to get the solution.\\
The solution column is the inverse of the standard basis vector you have just computed the solutions for.\\
\end{document}