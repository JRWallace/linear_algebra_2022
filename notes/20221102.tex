\documentclass{article}
\usepackage[utf8]{inputenc}
\usepackage{amsmath}
\usepackage{amsfonts}
\usepackage{amssymb}

\begin{document}
\section*{Nov 2 notes}\\
\\
Activity 3.3.13\\
Let there be a linear transformation with the standard matrix A, which of the following is equal to the dimension of the kernel T?\\
\\
The pivot columns tell us about the dimension of the image, the non-pivot columns tell us about the basis of the kernel. See Fact 3.3.11 for explanation re: why that is true. \\
\\
3.3.13 pivot columns tells us about the image, non-pivot columns is the basis of kernel, so the answer is B, the number of non-pivot columns.\\
D tells us about how much of the codomain we have spanned.\\
\\
See Observation 3.3.15 -- The rank nullity theoryem
the dim of domain t is the dim kernel dim=

1 column for every dim of input, so dimension of domain
every column is eeither a kernel or an image\\
\\
the dim of the image or the rank of a transformation and the rank of the kernel is the nullity\\
\\
Definition 3.4.1 - T is injective if its a linear transformation that doesnt make two vectors to the same vector -- 
\\
3.4.2 injective or not? - D\\
if we take in two different vectors, we still have the same output\\
3.4.3 injective or not? - B\\ 
why did we change from a to b -- for it to be injective it can't be equal -- b is more correct if we go back to the definition. The statement A is always true for linear transformations. \\
\\
For example:\\
\\
no matter what, if i start w two unique xy i'll get back two unique\\
\\
if something is injective does it need to have only pivot columns?\\
\\
Rebecca suggests we can't have an injective transformation if we go down a vector\\
\\
Injective and surjective arent musually exclusive, thigns can be either or both.
Surjectivity - T is surjective if every element of W is mapped to an element V.\\
Def 3.4.5 (b) not surjective example the set of vectors we get back only spand a 2d plane in R3, so we arent sujective\\
Not surjective bc we cant produce every vector in R3 -- bc we arent able to get a z value -- basically we span some plane in R3 but dont span the whole space.
\\
3.4.6 (a) \\
\\
what comes out is dependent on x and y, which is a -- the opposite is basically saying that no matter what you get out its dependent on z and has nothing to do w xy because weve set them to 0. If we plug in 00z in the original transformation, i'd always get back the zero vector.\\
\\
Observation 3.4.7 --- Injective map RREF has pivot in every column, surjective has a pivot in each row
\\
infinite solutions vs linear dependence\\
\\
If map goes from R3 to R3, and 
$\begin{bmatrix} 1&0&0 \\ 0&1&0 \\ 0&0&0 \end{bmatrix}$
is neither injective or surjective\\
\\
3.4.8 What can we conclude about the transformation if the kernel contains more than 1 vector?\\
Our answer is B\\
\\
not injective bc multiple things map to the same thing\\
it could be surjective but we dont know -- if we map r3 to r2 we might have multiple vectors that give us the zero vector, but you could still potentially span the whole space.\\
if we have more than 1 thing in our kernel, we have more than 1 free variable (check to make sure its true)\\
\\
Fact 3.4.9 - a linear transformation is injective when its trivial kernel -- e.g. T(0) is the only way to get 0 vector.\\
\\
Activity 3.4.10\\
\\
D, T is not surjective\\
\\
If the image is spanned by 4 vectors, it can't span R5 
\\
Fact 3.4.11 -- surjective - identical codomain and image\\
\\
next time activity 3.4.12 


\\

\\

\\
\\

\\
\\



\end{document}