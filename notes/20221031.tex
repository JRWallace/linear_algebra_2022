\documentclass{article}
\usepackage[utf8]{inputenc}
\usepackage{amsmath}
\usepackage{amsfonts}
\usepackage{amssymb}

\begin{document}
\section*{Oct 31 notes}\\
\\
Activity 3.3.4\\
Let T:$\mathbb{R}^{3} \rightarrow  \mathbb{R}^{2}$ be the linear transformation given by the standard matrix.\\
$T = 
\left(\left[\begin{matrix} 
x\\y\\z 
\end{matrix}\right]\right)$ = 
$\left[\begin{matrix} 
3x + 4y -z \\ x + 2y + z \end{matrix}\right]$
\\
a. set the three equations to 0\\
$3x+4y-z = 0$\\
$x+2y+z = 0$\\
\\
Computing the RREF\\
\begin{verbatim}
(Matrix(QQ,[
[3,4,-1,0],
[1,2,1,0]
])).rref()
[ 1  0 -3  0]
[ 0  1  2  0]
\end{verbatim}\\
\\
This means: \\
1x = 3z\\
y = -1z \\
z = z\\
\\
Activity 3.3.5\\
Let T:$\mathbb{R}^{3} \rightarrow  \mathbb{R}^{2}$ be the linear transformation given by the standard matrix.\\
$T = 
\left(\left[\begin{matrix} 
x\\y\\z\\w 
\end{matrix}\right]\right)$ = 
$\left[\begin{matrix} 
2x + 4y + 2z - 4w \\ 
-2x - 4y + z + w \\
3x + 6y -z - 4w \\
\end{matrix}\right]$
Again, compute RREF\\
\\
\begin{verbatim}
(Matrix(QQ,[
[2,4,2,-4],
[-2,-4,1,1],
[3,6,-1,-4]
])).rref()
[ 1  0 -3  0]
[ 0  1  2  0]
\end{verbatim}\\
-2,1,0,0
1 vector in our basis, kernel is 1 dimensional\\
\\
Activity 3.3.6 Which of these subspaces of R3 describes the set of all vectors that are the result of using t to transform R2 vectors?\\
Our answer is B bc we couldn't get anything in the z row because we can't get anything other than 0 in z row as a linear combination of T(xy) since it's always 0\\
\\
Let T:V$\rightarrow$W be a linear transformation, the image of R is an important subspace of W defined by\\
\\
Activity 3.3.8 \\
B is kernel, our answer is C bc we have x and y (not just a,a). 
\\
Activity 3.3.9\\
(Matrix(QQ,[
[3,4,7,1],
[-1,1,0,2],
[2,1,3,-1]
])).rref()
[ 1  0  1 -1]
[ 0  1  1  1]
[ 0  0  0  0]\\
A - not linearly independent
Im(t) is image of (T).
bc the standard matrix is the identity matrix, you can get any linear combination in R4\\
\\
Any vector can be rewritten as a linear combination \\
\\
The image of T is the span of the columns of A. Remove the vectors
creating non-pivot columns in RREF A to get a basis for the image.\\
we dont know if its a basis yet, but we do know it spans\\
\\
bc they are linear combinations of vectors of T it is linearly dependent.\\
\\
Observation 3.3.10 explains this a little more\\
Fact 3.3.11 the kernel is the solution set for the homogeneous system given the augmented matrix, use the coefficients of its free variables to get a basis for the kernel.\\
\\
THe image T is the span of the columns of A, remove the vectors creating non-pivots to get a basis for the image.\\
\\
3.3.13 pivot columns tells us about the image, non-pivot columns is the basis of kernel

\end{document}