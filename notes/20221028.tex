\documentclass{article}
\usepackage[utf8]{inputenc}
\usepackage{amsmath}
\usepackage{amsfonts}
\usepackage{amssymb}

\begin{document}
\section*{Oct 28 notes}\\
\\
Definition 3.2.7 - seems pretty critical for this whole section -- basically transpose the T: bits so they are applied to all columns, if they are the standard basis vectors you can use them for row operations.\\
\\
Activity 3.2.9\\
\\
Let the linear transformation be given by:\\
Turn the x,y,z into standard basis vectors\\
\\
x = [1,0,0], y = [0,1,0], z = [0,0,1]\\
Then we convert the system of equations to a matrix:\\
\\
$\left[ \begin{matrix} 1 & 0 & 3\\ 2 & -1 & -4  \end{matrix} \right] $
 \\


Fact 3.2.10: Because every linear map $T \colon \mathbb{R}^{m} \rightarrow \mathbb{R}^{n}$ has a linear combination of the variables in each component, and thus $T(\vec{e}_{i})$
yields exactly the coefficients of $x_{i}$, the standard matrix for $T$ is simply an ordered list of the coefficients of the $x_{i}$\\
\\
Fact 3.2.10: BC every linear map has a linear combination of the variables in each component, transformation of the standard basis vectors yields the coefficients of $x_{i}$  the standard basis matrix for T is an ordered list of coefficients in $x_{i}$\\
\\
Activity 3.2.11
$\left[ \begin{matrix} 3 & -2 & -1 \\ 5 & 5 & 2 \\ 0 & -2 & 1 \end{matrix} \right]$
(a) compute T [1,2,3]:\\
(b) compute T [x,y,z]:\\
\\
b first:\\
$\left[ \begin{matrix} 3x & -2y & -1z \\ 5x & 5y & 2z \\ 0x & -2y & 1z \end{matrix} \right]$\\
\\
then a:\\
$\left[ \begin{matrix} 3 & 4 & -3 \\ 5 & 10 & 6 \\ 0 & -4 & 3 \end{matrix} \right]$\\
\\
Then add them all together to get a vector:\\
$\left[ \begin{matrix} 4\\ 21 \\ -1 \end{matrix} \right]$\\
\\
\newpage
3.2.12\\
\\
$T_{1} \left( \left[ \begin{matrix} 
1\\2 
\end{matrix} \right] \right)$ 
$\left[\begin{matrix}
4 & 3 \\0 & -1 \\1 & 1\\3 & 0
\end{matrix}\right]$\\
\\
$\left[\begin{matrix}
4 & 6 \\0 & -2 \\1 & 2\\3 & 0
\end{matrix}\right]$ = 
$\left[\begin{matrix}
10 \\ -2 \\3\\3 \end{matrix}\right]$\\
\\
$T_{2} \left( \left[ \begin{matrix} 
1\\1\\0\\-3 
\end{matrix} \right] \right)$ 
$\left[\begin{matrix}
4 & 3 & 0 & -1 \\1 & 1 & 3 & 0
\end{matrix}\right]$\\
\\
$\left[\begin{matrix}
4 & 3 & 0 & 3 \\1 & 2 & 0 & 0 
\end{matrix}\right]$ = 
$\left[\begin{matrix}
10 \\3
\end{matrix}\right]$
\\
3.3.1 - answer is [0,0].\\
\\
the kernel of T is a subspace of V that yields the additive identity (zero).\\
\\
3.3.3 which of these subspaces describe kernel T - the set of all vectors that tranform into 0 vector.\\
Answer is A (bc c is part of a).\\
\\



\end{document}