\documentclass{article}
\usepackage[utf8]{inputenc}
\usepackage{amsmath}
\usepackage{amsfonts}
\usepackage{amssymb}

\begin{document}
\section*{Nov 4 notes from Rebecca }\\
\\
Activity 3.4.12\\
T is injective: A, C, D, H \\
T is surjective: B, E, F, G \\
\\
Note: If the transformation T is from Rn to Rn and T is found to be injective, T is also surjective. Likewise, if the transformation T is from Rn to Rn and T is found to be surjective, T is also injective. This is because if every column in a square matrix (because an  Rn to Rn transformation will have a square matrix) has a pivot, every row has a pivot, and vice versa (from statements D and E). \\
\\
Activity 3.4.14\\
C. Its standard matrix has more rows than columns, so T is not surjective. \\
Note: The standard matrix having more rows than columns is another way of saying that the transformation is increasing dimensions. Having more columns than rows would indicate that the transformation is decreasing dimensions. \\
\\
Activity 3.4.15\\
A. Its standard matrix has more columns than rows, so T is not injective. \\
\\
Activity 3.4.17\\
a. RREF A must have 4 pivot rows. \\
b. RREF A must have 4 pivot columns. \\
c. RREF A will be the 4-by-4 identity matrix. \\
Note: A transformation can only be both injective and surjective if the matrix T is square. \\
\\
Activity 3.4.18\\
True: A, B, C\\
False:\\
\\
Activity 3.4.20\\
D. T is bijective.\\ 
\\
Activity 3.4.21\\
D. T is neither injective nor surjective. \\
\\



\end{document}