\documentclass{article}
\usepackage[utf8]{inputenc}
\usepackage{amsmath}
\usepackage{amsfonts}
\usepackage{amssymb}

\begin{document}
\section*{Nov 9 notes}\\
Note: no notes for Nov 7 because it was an office hours session.\\
\\
\textbf{Observation 4.1.1}\\
If $T:\mathbb{R}^{n} \rightarrow \mathbb{R}^{m}$ and $S:\mathbb{R}^{m} \rightarrow \mathbb{R}^{k}$ are linear maps, then the composition map S ${\circ}$ T is a linear map from $\mathbb{R}^{n} \rightarrow \mathbb{R}^{k}$\\
\\
The outputs/inputs have to match in order to make a composition map. Activty 4.1.2 will work because we have an R2 intermediate.\\
\\
\textbf{Activity 4.1.2}\\
$T:\mathbb{R}^{3} \rightarrow \mathbb{R}^{2}$ is B = \begin{bmatrix} 2 & 1 & -3 \\ 5 & -3 & 4 \end{bmatrix}\\
\\
$S:\mathbb{R}^{2} \rightarrow \mathbb{R}^{4}$ is A = \begin{bmatrix} 1&2 \\ 0&1 \\ 3&5 \\ -1&-2 \end{bmatrix}\\
\\
What are the domain and codomain of the composition map S ${\circ}$ T?\\
\\
\textbf{Definition 3.1.2} says given a linear transformation T : V → W , V is called the domain of T and W is called the co-domain of T.\\
\\
The composition map is $\mathbb{R}^{3} \rightarrow \mathbb{R}^{4}$, so domain is $\mathbb{R}^{3}$ and co-domain is $ \mathbb{R}^{4}$  \\
\\
Therefore, the answer is C.\\
\\
\textbf{Activity 4.1.3}\\
$T:\mathbb{R}^{3} \rightarrow \mathbb{R}^{2}$ is B = \begin{bmatrix} 2 & 1 & -3 \\ 5 & -3 & 4 \end{bmatrix}\\
\\
$S:\mathbb{R}^{2} \rightarrow \mathbb{R}^{4}$ is A = \begin{bmatrix} 1&2 \\ 0&1 \\ 3&5 \\ -1&-2 \end{bmatrix}\\
\\
What size will the standard matrix of S ${\circ}$ T : $\mathbb{R}^{3} \rightarrow \mathbb{R}^{4}$ be? (rows x columns)\\
\\The composition map is $\mathbb{R}^{3} \rightarrow \mathbb{R}^{4}$, so T then S.\\
We work from left to right, so S ${\circ}$ T.\\
Could also just look at the size of the domains and co-domains. We are going from R3 to R4, so we go from R to L to get 4 by 3. Could also just think about 3 columns going in because there's 3 basis vectors for R3. We know we should have 4 rows bc we know at the end we'll be in R4, so each column will have 4 entries.\\
\\
What does this map do to standard basis vectors?
\\
$S {\circ} T (\vec{e}_{1}) = 
S \begin{bmatrix} 2 \\ 5 \end{bmatrix} =
2 \begin{bmatrix} 1 \\ 0 \\3 \\ -1 \end{bmatrix} + 
5 \begin{bmatrix} 2 \\ 1 \\ 5\\ -2 \end{bmatrix} = \begin{bmatrix} 2 \\ 0 \\6 \\ -2 \end{bmatrix} + 
\begin{bmatrix} 10 \\ 5 \\ 25\\ -10 \end{bmatrix} = \begin{bmatrix} 12 \\ 5 \\31 \\ -12 \end{bmatrix}$\\
\\
$S {\circ} T (\vec{e}_{2}) = 
S \begin{bmatrix} 1 \\ -3 \end{bmatrix} =
1 \begin{bmatrix} 1 \\ 0 \\3 \\ -1 \end{bmatrix} + 
-3 \begin{bmatrix} 2 \\ 1 \\ 5\\ -2 \end{bmatrix} = 
\begin{bmatrix} 1 \\ 0 \\3 \\ -1 \end{bmatrix} + 
\begin{bmatrix} -6 \\ -3 \\ -15\\ 6 \end{bmatrix} = 
\begin{bmatrix} -5 \\ -3 \\ -12 \\ 5 \end{bmatrix}$\\
\\
$S {\circ} T (\vec{e}_{3}) = 
S \begin{bmatrix} -3 \\ 4 \end{bmatrix} =
-3 \begin{bmatrix} 1 \\ 0 \\3 \\ -1 \end{bmatrix} + 
4 \begin{bmatrix} 2 \\ 1 \\ 5\\ -2 \end{bmatrix} = \begin{bmatrix} -3 \\ 0 \\-9 \\ 3 \end{bmatrix} + 
\begin{bmatrix} 8 \\ 4 \\ 20 \\ -8 \end{bmatrix} = \begin{bmatrix} 5 \\ 4 \\ 11 \\ -5 \end{bmatrix}$\\
\\
$S {\circ} T (\vec{e}_{4}) = 
S \begin{bmatrix} 0 \\ 0 \end{bmatrix} =
0 \begin{bmatrix} 1 \\ 0 \\3 \\ -1 \end{bmatrix} + 
0 \begin{bmatrix} 2 \\ 1 \\ 5\\ -2 \end{bmatrix} = \begin{bmatrix} 0 \\ 0 \\0 \\ 0 \end{bmatrix} + 
\begin{bmatrix} 0 \\ 0 \\ 0 \\ 0 \end{bmatrix} = \begin{bmatrix} 0 \\ 0 \\ 0 \\ 0 \end{bmatrix}$\\
\\
$\begin{bmatrix} 12 \\ 5 \\31 \\ -12 \end{bmatrix}$, 
$\begin{bmatrix} -5 \\ -3 \\ -12 \\ 5 \end{bmatrix}$, 
$\begin{bmatrix} 5 \\ 4 \\ 11 \\ -5 \end{bmatrix}$,
$\begin{bmatrix} 0 \\ 0 \\ 0 \\ 0 \end{bmatrix}$\\
\\
\textbf{Definition 4.1.5}\\
We define the product AB of a m × n matrix A and a n × k matrix B to be the m × k standard matrix of the composition map of the two corresponding linear functions.\\
\\
\textbf{Activity 4.1.6}\\
\\
$S:\mathbb{R}^{3} \rightarrow \mathbb{R}^{2}$ is 
A = \begin{bmatrix} -4 & -2 & 3 \\ 0 & 1 & 1 \end{bmatrix}\\
$T:\mathbb{R}^{2} \rightarrow \mathbb{R}^{3}$ is 
B = \begin{bmatrix} 2&3 \\ 1&-1 \\ 0&-1  \end{bmatrix}\\
\\
(a) Write the dimensions (rows × columns) for A, B, AB, and BA.\\
\\
A = 2 x 3, B = 3 x 2, AB = 2 x 2, BA = 3 x 3\\
In the case of AB, we are going B to A, which means we are going from R2 to R3, so 2 columns going in from R2 and 3 rows because we are in R3. \\
Then, we are going back to R2, so we are going back to 2 columns and 2 rows. \\
In the case of BA, we are going from A to B, which means 3 columns going in and 3 rows coming out, so 3x3.\\
\\Could also just look at the size of the domains and co-domains. We are going from R3 to R4, so we go from R to L to get 4 by 3. Could also just think about 3 columns going in because there's 3 basis vectors for R3. We know we should have 4 rows bc we know at the end we'll be in R4, so each column will have 4 entries.\\
\\
(b) Find the standard matrix AB of S ◦ T.\\
(c) Find the standard matrix BA of T ◦ S.\\
\\
$S {\circ} T (\vec{e}_{1}) = 
S \begin{bmatrix} -4\\0 \end{bmatrix} =
-4 \begin{bmatrix} 2 \\ 1 \\ 0 \end{bmatrix} + 
0 \begin{bmatrix} 3 \\ -1 \\ -1\end{bmatrix} = \begin{bmatrix} 2 \\ 0 \\6 \\ -2 \end{bmatrix} + 
\begin{bmatrix} 10 \\ 5 \\ 25\\ -10 \end{bmatrix} = \begin{bmatrix} 12 \\ 5 \\31 \\ -12 \end{bmatrix}$\\
\\
\\The composition map is $\mathbb{R}^{3} \rightarrow \mathbb{R}^{4}$, so T then S.\\
We work from left to right, so S ${\circ}$ T.

\textbf{Activity 4.1.8}\\
Let $B=\left[\begin{array}{ccc} 3 & -4 & 0 \\ 2 & 0 & -1 \\ 0 & -3 & 3 \end{array}\right]$ and $A=\left[\begin{array}{ccc} 2 & 7 & -1 \\ 0 & 3 & 2 \\ 1 & 1 & -1 \end{array}\right]\text{.}$
\begin{verbatim}
    B = [3 -4 0 ; 2 0 -1 ; 0 -3 3]
    A = [2 7 -1 ; 0 3 2 ; 1 1 -1]
    B*A
 6 9 -11
 3 13 -1
 3 -6 -9
\end{verbatim}
\end{document}