\documentclass{article}
\usepackage[utf8]{inputenc}
\usepackage{amsmath}
\usepackage{amsfonts}
\usepackage{amssymb}

\begin{document}
\section*{Oct 26 notes}\\

Activity 3.2.2.\\
Our answer is A $\begin{matrix} 6 \\ 3\end{matrix}$ because we can multiply $\begin{matrix} 1\\0\\0 \end{matrix}$\\
\\
For 3.2.3 you'd basically add $\begin{matrix} 2\\1 \end{matrix}$ and $\begin{matrix} -3\\2\ \end{matrix}$ and we'd get C $\begin{matrix} -1\\3 \end{matrix}$\\
\\
For 3.2.4 we'd basically multiply $\begin{matrix} 1\\0\\0 \end{matrix}$ transformation by -2 and $\begin{matrix} 0\\0\\1 \end{matrix}$ transformation by -3 and then add them together to get C $\begin{matrix} -4\\-2 \end{matrix}$ and $\begin{matrix}9\\-6 \end{matrix}$ so $\begin{matrix} 5 \\-8 \end{matrix}$ \\
\\
For 3.2.5 - any of them bc you can get the middle value of T thru linear combination, we'll be able to span the set of all vectors of T, if you take the RREF you'd see thats true -- with B they are a basis.
\\
Fact 3.2.6 -- Consider any basis for V -- since every vector can be written as a linear combo of the basis vectors, we may compute T(v) as follows:

Therefore, any linear transformation of T:V W can be dfined by describing the values of T(b_{i})

So just describing the transformation in terms how it effects the basis vectors will completely determine the transformation.\\
\\
Definition 3.2.7 \\
A linear transformation is determined by its action on the standard basis... so its convenient to store this info in an m x n matrix, callled the standard matrix of T\\
Activity 3.2.8\\
the e1 implies that its always top left, first pivot.. so, just like what we did in def 3.2.7, you'd combine those 4 vectors in a matrix and then we'd get the standard basis.\\

$\left[ 
\begin{matrix}

\end{matrix}
\right]$



\end{document}